% Notes Template
% Template version: v0.4
% https://github.com/rnsavinelli/notes-template
%
%!TEX encoding = UTF-8 Unicode

% Document-class options:
\documentclass[letter]{article}

% Package includes and formatting settings -----------------------------------
% ----------------------------------------------------------------------------

% LaTeX Font encoding - 8-bit fonts
\usepackage[T1]{fontenc}
\usepackage[utf8]{inputenc}
\usepackage[sc, osf]{mathpazo}
\usepackage[spanish]{babel}

\usepackage{hyperref}

% These settings should be modified by the user
\def\authorsname{Mr./Mrs. Author}
\def\authorswebsite{}
\def\course{\LaTeX}
\def\courseyear{}
\def\thetitle{Title}
\def\thesubtitle{Subtitle}
\def\university{}

% PDF Settings
\hypersetup{
    pdfauthor = {\authorsname},
    pdfkeywords = {-},
    pdftitle = {\thetitle \\ {\large \thesubtitle}},
    pdfsubject = {-},
}

% HREF settings
\hypersetup{
    colorlinks = true,
    urlcolor = blue,
    linkcolor = black
}

% LaTeX' own graphics handling
\usepackage{graphicx}
\graphicspath{ {./images/} }

% AMS-LaTeX extensions for mathematical typesetting.
\usepackage{amsmath,amsthm,amsfonts,amssymb,mathrsfs}

% Document format and settings -----------------------------------------------
% ----------------------------------------------------------------------------
\usepackage[left=1in, right=1in, bottom=1.25in, top=1.25in]{geometry}

% Better spacing between lines
\usepackage{parskip}

\setlength\parindent{0em}
%\pagenumbering{roman}

\usepackage{chngcntr}
\counterwithin*{equation}{section}

% Title, section, and topic formatting
\usepackage{titlesec}
\usepackage[labelfont=bf]{caption}

% \today commands
\usepackage{datetime}

% Content of the document ----------------------------------------------------
% ----------------------------------------------------------------------------

\begin{document}

\title{\thetitle \\ {\large \thesubtitle}}
\author{\authorsname}
\date{\today}

\maketitle

% Include as many files as needed. This approach is taken to reduce the amount
% of information contained inside this very same document. It is not mandatory
% to use external files to add your content, but it is advisable.
%
% \input is preferred over \include since the latter automatically appends a
% page break. Deppending of your use case you might prefer one over the other.

%!TEX root = notex.tex

%\section{}

%\subsection{}

%!TEX root = notex.tex

%\section{}

%\subsection{}


% Closing of the document ----------------------------------------------------
% ----------------------------------------------------------------------------
% Comment the commands below to re-enable section,
% subsection and subsubsection numbering.

\titleformat{\section}{\bfseries\Large}{}{0em}{}
\titleformat{\subsection}{\bfseries\large}{}{0em}{}
\titleformat{\subsubsection}{\bfseries\normalsize}{}{0em}{}

% Bibliography ---------------------------------------------------------------
% ----------------------------------------------------------------------------

%\newpage
%\bibliographystyle{plain}
%\bibliography{bib/uni}

% Appendices -----------------------------------------------------------------
% ----------------------------------------------------------------------------

%\newpage
%%!TEX root = notex.tex

%\section{Appendix A}


% About Section --------------------------------------------------------------
% ----------------------------------------------------------------------------

%\newpage
%%!TEX root = notex.tex

%\section{About this document}

%Last updated: {\today}.


\end{document}

% End of the document --------------------------------------------------------
% ----------------------------------------------------------------------------
